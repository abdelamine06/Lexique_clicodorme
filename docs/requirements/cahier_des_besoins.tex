\documentclass[a4paper, 12pt]{article}
\usepackage[french]{babel}
\usepackage[utf8]{inputenc}
\usepackage[left=2cm,right=2cm,top=3cm,bottom=2cm]{geometry}
\usepackage{graphicx}
\usepackage{xcolor}
\usepackage{color}
\usepackage{ulem}
\usepackage{setspace}
\usepackage{fancyhdr}
\usepackage[section,above,below]{placeins}
\usepackage[final]{pdfpages}
\pagestyle{fancy}
\fancyhf{}


\usepackage{footnote}

\begin{document}

\begin{titlepage}
\centering
{\scshape\LARGE Université de Bordeaux \par}
{\scshape\Large Master 1 Informatique  \par}
\vspace{2cm}

{\Huge\bfseries Projet de Programmation\par}
\vspace{2cm}
{\Huge\bfseries Le Clicodrome pour Lexique Electronique \par}
\vspace{1cm}
{\Large\bfseries par Lionel CLEMENT \par}
\vspace{3cm}
{\Large\bfseries Cahier des Besoins\par}

\vspace{1cm}
réalisé par \par
MEHDAOUI  \textsc{Abdelamine} \par
ROUSSEL  \textsc{Damien} \par
EL GUERCH  \textsc{Souhail} \par
DAOUDI \textsc{Yassir } \par
DIALLO  \textsc{Abdoul Ghadiri} \par

\vspace{3cm}
{\large Enseignant responsable : Philippe NARBEL \par}
\vspace{0.5cm}
{\large\bfseries \today\par}

\end{titlepage}
	

\newpage
\tableofcontents

\newpage\section{Introduction}
\subsection{Contexte}
Dans le cadre de notre Cursus de Master 1 Informatique à l'Université de bordeaux, nous devons au compte de l'unité d'enseignement Projet de Programmation réaliser par groupe de cinq étudiants un projet de développement informatique. Nous travaillons avec Monsieur \textbf{Lionel CLEMENT} notre client et monsieur \textbf{Simon ARCHIPOFF} chargé de l'encadrement de nos travaux dirigés.

\subsection{Présentation du projet}

{\textbf{Lionel CLEMENT}, spécialisé dans le domaine de la linguistique formelle et du traitement automatique des langues est un enseignant chercheur au Laboratoire Bordelais de Recherche en Informatique(\textbf{LABRI}). Il a réalisé avec \textbf{Benoit SAGOT} chercheur à l'Institut National de Recherche en Informatique et Automatique(\textbf{INRIA})  le Lexique électronique des formes fléchies du français, qu'ils ont appelé \textbf{LeFFF}.\par}
{Le français est une langue, une langue peut-être définie comme un ensemble de signes oraux et écrits qui permettent à un groupe d'individus de communiquer. Les mots qui forment le français se repartissent en neuf classes: les noms, les déterminants, les adjectifs qualificatifs, les pronoms, les verbes, les adverbes,les prépositions, les conjonctions de coordination et les interjections. Partant de cette classification on peut dire que les formes fléchies d'un mot correspondent aux différentes déclinaisons de ce mot (sa conjugaison pour un verbe, son singulier et son pluriels pour certaines catégories de mots, ses synonymes,...
Dans le \textbf{LAROUSSE}, un \textit{\bf lexique} est un dictionnaire spécialisé et généralement succinct concernant un domaine particulier de la connaissance, c'est une ressource complexe constituée de \textit{lexèmes}, de \textit{vocables}, de \textit{catégories} et \textit{sous-catégories syntaxiques}, de \textit{catégories grammaticales}, de \textit{règles de flexion}, de \textit{valence}, de \textit{phraséologie}, de \textit{fonctions lexicales},...\par}

{Habituellement un dictionnaire est utilisé pour trouver la signification d'un mot qu'on connaît et il n'offre pas la possibilité de faire l'inverse, c'est à dire qu'à partir d'explications ou d'une liste de mots trouver le mot correspondant (un gain de temps considérable pour tous). Aujourd'hui \textbf{LeFFF} est un fichier texte et il n'existe aucun outil qui en facilite l'accès et la compréhension encore moins sa modification ou son enrichissement. Le \textbf{LeFFF} actuel souffre d'un manque de traçabilité et n'offre aucune garantie pour contrôler les modifications\par}

{L'objectif de ce projet est donc d'implémenter une application Web liée à une base de données(qui contiendra les mots du lexique) pour faciliter les interactions avec le lexique (consultation, ajout, suppression,...).\par}


\subsection{Description de l'existant et choix de conception}
{Nous utiliserons le \textbf{LeFFF}, un fichier contenant une base très importante de mots de français implémentant la structure du \textbf{FFF}. Ce fichier nous permettra d'avoir une base de mots, ainsi que d'avoir une structure rendant plus simple a implémenter les algorithmes de recherche du \textit{\bf lexique} en utilisant des \textit{transducteurs} et des \textit{unificateurs}.\par}

\section{Besoins fonctionnels}
Lors de nos différents échanges avec le client nous avons pu identifier ses besoins pour ce projet. En effet, la réalisation de ce projet nécessite plusieurs parties:
\begin{itemize}
\item Le développement d'une \textbf{application Web} permettant aux utilisateurs de facilement interagir avec le \textbf{lexique}
\item La mise en place d'une base de données contenant les différents mots ou expressions du \textbf{lexique}
\end{itemize}


\subsection{Rechercher un lexique}

{Nous offrirons la possibilité de rechercher un \textbf{lexique} de la même manière qu'un dictionnaire, mais aussi en permettant de le chercher en fonction de son \textbf{lemme} et \textbf{lexème} (sa définition ou synonyme).\par}

\subsection{Ajouter un mot ou une expression au lexique}

{Nous offrirons la possibilité d'ajouter un mot ou expression au \textbf{lexique} en rentrant les données de la structure du \textbf{FFF} pour celui-ci.\par}

\subsection{Supprimer un mot ou une expression du lexique}
{Nous offrirons la possibilité de supprimer un mot ou expression du \textbf{lexique}.\par}

\subsection{Modifier un lexique}
 
{Nous offrirons la possibilité de modifier un mot ou expression du \textbf{lexique} en modifiant un ou plusieurs champs de la structure du \textbf{FFF} de ce mot.\par}
 
\subsection{Gérer les rôles}

{Ici nous avons plusieurs possibilités, la première est que seuls des personnes autorisées peuvent ajouter, modifier ou supprimer des mots du \textbf{lexique}. La deuxième est une forme collaborative par tout utilisateur qui pourront ajouter, modifier ou supprimer des mots, cette solution apportera des difficultés car il faudra pouvoir gérer l'historique des modifications pour pouvoir revenir dans un état antérieur.\par}

Comme tout site internet le nôtre aura différents \textbf{types d'utilisateurs}. Pour contrôler les interactions de ces utilisateurs et garantir l'intégrité des données un système de rôle sera mis en place sur le site.
 
 \subsection{Exporter le LeFFF}

{Nous offrirons la possibilité d'exporter le \textbf{lexique} sélectionné avec un \textbf{filtre} afin que l'utilisateur puisse le \textbf{télécharger}\par}

\section{Les besoins non fonctionnels}

\subsection{Sécurité}
{Le système doit offrir un \textbf{accès personnalisé} pour nos utilisateurs (chacun à son rôle), de plus doit utiliser des \textbf{protocoles de sécurité} ou des certificats \textsc{Https} }

\subsection{Format de fichier}
Nous offrirons la possibilité de gérer le téléchargement du \textbf{LeFFF} ou d'une partie du \textbf{lexique} sous format \textbf{txt}.

\subsection{Performances}

{Nous devrons faire en sorte à ce que la \textbf{base de données} soit \textbf{optimisée} afin qu'une recherche ou exportation par de multiple personnes en même temps ne prenne par un temps trop grand.\par}
Nous sommes dans l'obligation de manipuler la \textbf{gestion de la mémoire} d'une manière très \textbf{efficace} dans le but d'assurer le fonctionnement des besoins fonctionnels dans tous les scénarios d'utilisation possible du logiciel. 

\subsection{Ergonomie}
L’ergonomie touche à la \textbf{qualité} et au \textbf{confort d’utilisation} de la navigation web, dans notre cas le client nous a donné une liberté dans le choix de l'agencement de la page, la police, les couleurs, le design. 

\section{Planning du projet}

\section{Outils de développement}

\end{document} 








}