\section{Besoins fonctionnels}
Lors de nos différents échanges avec le client nous avons pu identifier ses besoins pour ce projet. En effet, la réalisation de ce projet nécessite plusieurs parties:
\begin{itemize}
\item Le développement d'une \textbf{application Web} permettant aux utilisateurs de facilement interagir avec le \textbf{lexique}
\item La mise en place d'une base de données contenant les différents mots ou expressions du \textbf{lexique}
\end{itemize}


\section{Les besoins non fonctionnels}

\subsection{Sécurité}
{Le système doit offrir un \textbf{accès personnalisé} pour nos utilisateurs (chacun à son rôle), de plus doit utiliser des \textbf{protocoles de sécurité} ou des certificats \textsc{Https} }

\subsection{Format de fichier}
Nous offrirons la possibilité de gérer le téléchargement du \textbf{LeFFF} ou d'une partie du \textbf{lexique} sous plusieurs format au choix  \textbf{txt},\textbf{xml}, \textbf{ilex} : pour le radical des mots c'est à dire une version de base pas compilé du leFFF intensif, \textbf{tlex} : format compilé niveau extensif.

\subsection{Performances}

{Nous devrons faire en sorte à ce que la \textbf{base de données} soit \textbf{optimisée} afin qu'une recherche ou exportation par de multiple personnes en même temps ne prenne par un temps trop grand.\par}
Nous sommes dans l'obligation de manipuler la \textbf{gestion de la mémoire} d'une manière très \textbf{efficace} dans le but d'assurer le fonctionnement des besoins fonctionnels dans tous les scénarios d'utilisation possible du logiciel. 

\subsection{Ergonomie}
L’ergonomie touche à la \textbf{qualité} et au \textbf{confort d’utilisation} de la navigation web, dans notre cas le client nous a donné une liberté dans le choix de l'agencement de la page, la police, les couleurs, le design. 

\subsection{Compléxité}
Lors de la recherche d'un mot dans la base de données , l'algorithme de recherche devra parcourir tous les mots du \textbf{LeFFF}. En effet, le fait de parcourir tout le fichier \textbf{LeFFF} , rendra la recherche d'un mot ou d'une expression coûteux en temps , il faudra donc éviter d'utiliser  un algorithme qui aura une compléxité exponentielle. (la non maitrise du temps d'attente)Du fait d'un temps d'attente non maîtrisé et qui risque de rendre l'utilisation du site web peu ergonomique. Afin de répondre à cette problématique, privilégier un parcours en profondeur qui aura une compléxité linaire.
La complexité sera une grande partie dans la réalisation du projet 
