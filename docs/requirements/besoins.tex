\section{Besoins fonctionnels}
Lors de nos différents échanges avec le client nous avons pu identifier ses besoins pour ce projet. En effet, la réalisation de ce projet nécessite plusieurs parties:

\subsection{Partie Langue}
Mettre en place une base de données qui utilise le niveau intensif du \textbf{LeFFF} pour former les différents mots du \textbf{lexique}.

\subsubsection{Génération des formes fléchies - Importance 5/5}
Ce besoin est le \textbf{besoin principal} de notre projet. Si notre base de données devait contenir en plus des mots du lexique, toutes les expressions possibles qui vont avec, le projet aurait été vraiment très simple (c'est à dire un site internet qui permet d'enregistrer et de manipuler des informations). En effet, une \textbf{génération dynamique} des formes fléchies permettra une \textbf{réduction} considérable de la \textbf{taille du lexique} à enregistrer dans la base de données. Pour remédier à ce problème, il nous faut donc mettre en place un outil qui respecte les \textbf{règles linguistique} pour générer les formes fléchies.

\subsubsection{Base de Données - Importance 5/5}


{Nous allons utiliser une \textbf{base de données relationnelle}. A première vue il aurait été plus judicieux d'utiliser une base de données non relationnelle du fait du nombre volumineux de données à traiter. Néanmoins le choix d'utiliser ici une base de données relationnelle nous permettra une plus \textbf{simple} gestion des données en particulier pour la recherche. Afin \textbf{d'optimiser} la \textbf{recherche} nous utiliserons une ou plusieurs de ces 3 solutions : 
\begin{itemize}
\item Des \textbf{fichiers triés}, rapide pour la recherche (recherche dichotomique), mais lents pour l'insertion et la suppression.
\item Des \textbf{fichiers hachés} efficaces pour les sélections avec égalité.
\item Des \textbf{index} (pour chaque lettre de l'alphabet) pour permettre d'améliorer certaines opérations sur un fichier.
\end{itemize}\par}

\subsubsection{Rechercher un lexique - Importance 5/5}

{Nous offrirons la possibilité de rechercher un \textbf{lexique} de la même manière qu'un dictionnaire, mais aussi en permettant de le chercher en fonction de son \textbf{lemme} et \textbf{lexème} (sa définition ou synonyme). Par exemple l'entrée en recherche "contents", nous pourrons retrouver en sortie "content", "heureux", "être à la fête", "être aux anges", "être gai comme un pinson",... Par la suite, nous pourrons afficher toutes les formes possibles pour chacun des résultats retournés comme par exemple singulier, pluriel, masculin, féminin ou temps conjugués pour les verbes. Un exemple de test réalisable serait de rechercher un mot et de refaire une recherche avec l'un des résultats obtenus pour comparer les deux recherches.\par}

\begin{figure}[ht]
    \centering
    \includegraphics[scale=0.5]{exemple.png}
    \caption{Scénario d'utilisation pour la recherche d'un mot }
\end{figure}
\newpage

\subsubsection{Transducteur - Importance 5/5}


La composition des \textbf{transducteurs} nous permettra de construire un \textbf{lexique} spécifique à la nature de l'entrée, nous réaliserons des \textbf{transducteurs} qui vont \textbf{conjuguer} des verbes, d'autres qui vont \textbf{transformer} du masculin au féminin, du singulier au pluriel, ou \textbf{inversement}. Les algorithmes de construction des transducteurs pourront prendre en compte des \textbf{filtres} afin d'affiner la recherche et la rendre plus rapide.Chaque \textbf{arcs} correspondra à une \textbf{spécification} du mot (symboles signifiant la nature et la morphologie d'un mot).

{Il faut aussi mettre en place un \textbf{transducteur} qui aura pour fonction de \textbf{récupérer} le \textbf{radical} d'un mot fourni pour se faire nous utiliserons plusieurs automates finis afin de détecter sa catégorie grammaticale, pour ensuite connaître sa forme dans le but de retourner son radical.\par}

{Inversement, à partir d'un autre transducteur, celui ci devra retourner toutes ses formes.\par}

\begin{figure}[ht]
    \centering
    \includegraphics[scale=0.5]{transducteur.png}
    \caption{transducteur }
\end{figure}

\newpage

\subsubsection{Unificateur - Importance 5/5}{
\textbf{L'unificateur} est l'algorithme qui va nous permettre de \textbf{lier} plusieurs \textbf{mots} afin d'en tirer le \textbf{sens}, pour ce faire, il faut faire \textbf{l'union des informations} sur ces mots. S'il y a une \textbf{contradiction} avec ces informations, alors la phrase est \textbf{mal formée} et dans ce cas aucun sens ne sera retourné, sinon on pourra voir quel est le sens de l'expression formée.\par}

\begin{figure}[ht]
    \centering
    \includegraphics[scale=0.4]{unificateur.png}
    \caption{Unificateur }
\end{figure}

{Les mots utilisés sur \textbf{l'unificateur} utiliserons une \textbf{sous-structure commune} afin de faciliter et de rendre sensée cette opération, puis à l'aide de la \textbf{théorie de substitution des ensembles}, nous pourrons \textbf{optimiser} la construction de nos formules. Celles-ci nous permettrons par la suite d'\textbf{améliorer} la \textbf{rapidité} de l'algorithme de recherche, ainsi que la \textbf{mémoire} utilisée.\par}


\subsection{Partie Web}
Le développement d'une \textbf{application Web} permettant aux utilisateurs de facilement interagir avec le \textbf{lexique}.

\subsubsection{Ajouter un mot au lexique - Importance 4/5}

{{Nous offrirons la possibilité d'ajouter un mot au \textbf{lexique} en rentrant les données de la structure du \textbf{FFF}. Seules les \textbf{radicaux} des mots seront a rajouter, les différentes formes seront automatiquement générées par les transducteurs. Pour ajouter un mot, nous devrons entrer des informations sur celui-ci comme son sens et sa catégorie grammaticale, pour pouvoir ensuite être reconnu par l'algorithme de recherche demandé ci-dessus. Un exemple de test réalisable serait de voir si on trouve bien les éléments rajoutés avec la recherche de lexique.\par}

\subsubsection{Modifier un mot - Importance 4/5}
 
{Nous offrirons la possibilité de modifier un mot du \textbf{lexique} en modifiant un ou plusieurs champs de la structure du \textbf{FFF} de ce mot utilisés lors de l'ajout de celui-ci. Cette fonction permettra entre autre de corriger des erreurs d'ajout dans le lexique comme par exemple faute d'orthographe, mauvaise catégorisation,... Un changement de sens devra être répercuté dans tout le lexique du fait de sa génération automatique. Il sera possible de tester cette fonctionnalité en recherchant l'élément modifié directement et de constater ses changements (toutes les possibilités de changement seront donc a tester).\par}
 
\subsubsection{Supprimer un mot du lexique - Importance 1/5}
{Nous offrirons la possibilité de supprimer un mot du \textbf{lexique}, afin de les enlever de toute la base de données utilisée, et donc de faire en sorte à ce que l'algorithme de recherche ne le trouve plus. Cette fonction ne sera normalement que très peu utilisée car l'ajout et la modification du lexique est contrôlée par une gestion des rôles, rendant cette fonctionnalité peu utile. Un exemple de test réalisable serait de faire une recherche et par la suite de supprimer un des résultats, pour enfin réitérer la recherche sur tous les résultats pour vérifier que l'élément supprimé est bien supprimé.\par}
 
\subsubsection{Gérer les rôles - Importance 2/5}
    
Nous mettrons en place un système à trois \textbf{types d'utilisateurs}:
\begin{itemize}
\item \textbf{Utilisateur}: peut \textbf{s'inscrire, valider, supprimer} son compte, et peut \textbf{consulter} et extraire le lexique.
\item \textbf{Contributeur}: peut faire ce que fait un utilisateur, ainsi que  \textbf{modifier} le lexique.
\item \textbf{Administrateur}: peut faire ce que fait un modérateur, ainsi que \textbf{gérer les droits} des autres utilisateurs (faire passer certains utilisateurs contributeurs et vis versa).
\end{itemize}
{Cette façon de gérer les rôles reste simple mais efficace qui permettra de limiter au maximum les erreurs lors d'ajout d'éléments dans le \textbf{lexique}.\par}

\begin{figure}[ht]
    \centering
    \includegraphics[scale=0.5]{role.png}
    \caption{Gestion des rôles des utilisateurs }
\end{figure}
\newpage
    


\subsubsection{Exporter le LeFFF - Importance 3/5}

{Nous offrirons la possibilité d'exporter le \textbf{lexique} sélectionné à l'aide d'un \textbf{filtre} pour que l'utilisateur puisse \textbf{télécharger} un document contenant tous les détails de la recherche, il sera au format \textbf{Intensif} du \textbf{LeFFF}. En effet, celui-ci par rapport au format extensif permet une plus grande optimisation du fait de leur taille. Deux choix de formats seront possible, soit \textbf{xml}, soit \textbf{ilex}. Pour ce qui est des \textbf{filtres}, nous pourrons en avoir plusieurs types comme une \textbf{catégorie} de mot ou un sens (verbe, nom, adjectif,..., singulier, pluriel,..., définition, lexique,...). Il sera possible de tester cette fonction en exportant sans filtre le \textbf{LeFFF} au format ilex et par la suite vérifier que le fichier téléchargé est bien identique au \textbf{LeFFF}.\par}

\newpage
\section{Les besoins non fonctionnels}

\subsection{Sécurité}
{Le système doit offrir un \textbf{accès personnalisé} pour nos utilisateurs (chacun a son rôle), de plus doit utiliser des \textbf{protocoles de sécurité} ou des certificats \textsc{Https} }

\subsection{Format de fichier}
Nous offrirons la possibilité de gérer le téléchargement du \textbf{LeFFF} ou d'une partie du \textbf{lexique} sous plusieurs format au choix  \textbf{txt},\textbf{xml}, \textbf{ilex} : pour le radical des mots c'est à dire une version de base pas compilé du leFFF intensif, \textbf{tlex} : format compilé niveau extensif.

\subsection{Performances}

{Nous devrons faire en sorte à ce que la \textbf{base de données} soit \textbf{optimisée} afin qu'une recherche ou exportation par de multiple personnes en même temps ne prenne par un temps trop grand.\par}
Nous sommes dans l'obligation de manipuler la \textbf{gestion de la mémoire} d'une manière très \textbf{efficace} dans le but d'assurer le fonctionnement des besoins fonctionnels dans tous les scénarios d'utilisation possible du logiciel. 

\subsection{Ergonomie}
L’ergonomie touche à la \textbf{qualité} et au \textbf{confort d’utilisation} de la navigation web, dans notre cas le client nous a donné une liberté dans le choix de l'agencement de la page, la police, les couleurs, le design. Nous avons choisi de concevoir un site web simple qui sera d'une utilisation facile et efficace par tous types de visiteurs, en utilisant des instructions claires et basiques.

\subsection{Complexité}
Lors de la recherche d'un mot dans la base de données , l'algorithme de recherche devra parcourir la base de données. En effet, le fait de parcourir naïvement toute la base de données, rendra la recherche d'un mot ou d'une expression coûteux en terme de temps et de mémoire. Il faudra donc \textbf{éviter} d'utiliser un algorithme qui aura une \textbf{complexité exponentielle}. Du fait d'un temps d'attente non maîtrisé et qui risque de rendre l'utilisation du site web peu plaisante. Afin de répondre à cette problématique, privilégier un parcours rapide et intelligent qui aura une \textbf{complexité au plus polynomiale}.
La complexité sera une grande partie dans la réalisation du projet. 

