\section{Besoins fonctionnels}
Lors de nos différents échanges avec le client nous avons pu identifier ses besoins pour ce projet. En effet, la réalisation de ce projet nécessite plusieurs parties:
\begin{itemize}
\item Le développement d'une \textbf{application Web} permettant aux utilisateurs de facilement interagir avec le \textbf{lexique}
\item La mise en place d'une base de données contenant les différents mots ou expressions du \textbf{lexique}
\end{itemize}


\subsection{Rechercher un lexique}

{Nous offrirons la possibilité de rechercher un \textbf{lexique} de la même manière qu'un dictionnaire, mais aussi en permettant de le chercher en fonction de son \textbf{lemme} et \textbf{lexème} (sa définition ou synonyme).\par}

\subsection{Ajouter un mot ou une expression au lexique}

{Nous offrirons la possibilité d'ajouter un mot ou expression au \textbf{lexique} en rentrant les données de la structure du \textbf{FFF} pour celui-ci.\par}

\subsection{Supprimer un mot ou une expression du lexique}
{Nous offrirons la possibilité de supprimer un mot ou expression du \textbf{lexique}.\par}

\subsection{Modifier un lexique}
 
{Nous offrirons la possibilité de modifier un mot ou expression du \textbf{lexique} en modifiant un ou plusieurs champs de la structure du \textbf{FFF} de ce mot.\par}
 
\subsection{Gérer les rôles}

{Ici nous avons plusieurs possibilités, la première est que seuls des personnes autorisées peuvent ajouter, modifier ou supprimer des mots du \textbf{lexique}. La deuxième est une forme collaborative par tout utilisateur qui pourront ajouter, modifier ou supprimer des mots, cette solution apportera des difficultés car il faudra pouvoir gérer l'historique des modifications pour pouvoir revenir dans un état antérieur.\par}

Comme tout site internet le nôtre aura différents \textbf{types d'utilisateurs}. Pour contrôler les interactions de ces utilisateurs et garantir l'intégrité des données un système de rôle sera mis en place sur le site.
 
 \subsection{Exporter le LeFFF}

{Nous offrirons la possibilité d'exporter le \textbf{lexique} sélectionné avec un \textbf{filtre} afin que l'utilisateur puisse le \textbf{télécharger}\par}

\section{Les besoins non fonctionnels}

\subsection{Sécurité}
{Le système doit offrir un \textbf{accès personnalisé} pour nos utilisateurs (chacun à son rôle), de plus doit utiliser des \textbf{protocoles de sécurité} ou des certificats \textsc{Https} }

\subsection{Format de fichier}
Nous offrirons la possibilité de gérer le téléchargement du \textbf{LeFFF} ou d'une partie du \textbf{lexique} sous format \textbf{txt}.

\subsection{Performances}

{Nous devrons faire en sorte à ce que la \textbf{base de données} soit \textbf{optimisée} afin qu'une recherche ou exportation par de multiple personnes en même temps ne prenne par un temps trop grand.\par}
Nous sommes dans l'obligation de manipuler la \textbf{gestion de la mémoire} d'une manière très \textbf{efficace} dans le but d'assurer le fonctionnement des besoins fonctionnels dans tous les scénarios d'utilisation possible du logiciel. 

\subsection{Ergonomie}
L’ergonomie touche à la \textbf{qualité} et au \textbf{confort d’utilisation} de la navigation web, dans notre cas le client nous a donné une liberté dans le choix de l'agencement de la page, la police, les couleurs, le design. 
