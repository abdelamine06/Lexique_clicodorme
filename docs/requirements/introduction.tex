
\newpage
\section{Introduction}
\subsection{Contexte}
Dans le cadre de notre Cursus de Master 1 Informatique à l'Université de bordeaux, nous devons au compte de 
l'unité d'enseignement Projet de Programmation réaliser par groupe de cinq étudiants un projet de développement
 informatique. Nous travaillons avec Monsieur \textbf{Lionel CLEMENT} notre client et monsieur \textbf{Simon ARCHIPOFF}
 chargé de l'encadrement de nos travaux dirigés.

\subsection{Présentation du projet}

{\textbf{Lionel CLEMENT}, spécialisé dans le domaine de la linguistique formelle et du traitement automatique des langues
 est un enseignant chercheur au Laboratoire Bordelais de Recherche en Informatique(\textbf{LABRI}). Il a réalisé avec 
 \textbf{Benoit SAGOT} chercheur à l'Institut National de Recherche en Informatique et Automatique(\textbf{INRIA})  le Lexique
  électronique des formes fléchies du français, qu'ils ont appelé \textbf{LeFFF}.\par}
{Le français est une langue, une langue peut-être définie comme un ensemble de signes oraux et écrits qui permettent à un groupe
 d'individus de communiquer. Les mots qui forment le français se repartissent en neuf classes: les noms, les déterminants, les adjectifs
  qualificatifs, les pronoms, les verbes, les adverbes,les prépositions, les conjonctions de coordination et les interjections. Partant 
  de cette classification on peut dire que les formes fléchies d'un mot correspondent aux différentes déclinaisons de ce mot (sa conjugaison 
  pour un verbe, son singulier et son pluriels pour certaines catégories de mots, ses synonymes,...
Dans le \textbf{LAROUSSE}, un \textit{\bf lexique} est un dictionnaire spécialisé et généralement succinct concernant un domaine 
particulier de la connaissance, c'est une ressource complexe constituée de \textit{lexèmes}, de \textit{vocables}, de \textit{catégories} 
et \textit{sous-catégories syntaxiques}, de \textit{catégories grammaticales}, de \textit{règles de flexion}, de \textit{valence}, de 
\textit{phraséologie}, de \textit{fonctions lexicales},...\par}

{Habituellement un dictionnaire est utilisé pour trouver la signification d'un mot qu'on connaît et il n'offre pas la possibilité de faire
 l'inverse, c'est à dire qu'à partir d'explications ou d'une liste de mots trouver le mot correspondant (un gain de temps considérable pour tous).
  Aujourd'hui \textbf{LeFFF} est un fichier texte et il n'existe aucun outil qui en facilite l'accès et la compréhension encore moins sa 
  modification ou son enrichissement. Le \textbf{LeFFF} actuel souffre d'un manque de traçabilité et n'offre aucune garantie pour contrôler 
  les modifications\par}

{L'objectif de ce projet est donc d'implémenter une application Web liée à une base de données(qui contiendra les mots du lexique) pour faciliter 
les interactions avec le lexique (consultation, ajout, suppression,...).\par}


\subsection{Description de l'existant et choix de conception}
{Nous utiliserons le \textbf{LeFFF}, un fichier contenant une base très importante de mots de français implémentant la structure du \textbf{FFF}. 
Ce fichier nous permettra d'avoir une base de mots, ainsi que d'avoir une structure rendant plus simple a implémenter les algorithmes de recherche
 du \textit{\bf lexique} en utilisant des \textit{transducteurs} et des \textit{unificateurs}.\par}